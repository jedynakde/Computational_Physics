\documentclass{article}
\usepackage{amsmath}
\usepackage{graphicx}
\usepackage{listings}
\newcommand{\includecode}[2][c]{\lstinputlisting[caption=#2, escapechar=, language=#1]{#2}}

\begin{document}
\lstset{language=C} 
\title{Introducing Dynamic Walls in an Integer Lattice Gas Simulation}
\author{David Jedynak}
\maketitle
\begin{abstract}
This proposal shall provide a description of the dynamic walls to be added to an integer lattice gas simulation. Tasks will be layed out so  a group of more than one individual may work on this project collaboratively. Descriptions of milestones and methods to be used to achieve them will be defined. A timeline of the project progression will be defined.
\end{abstract}
\section{Problem Description}
The lattice gas code simulation is currently unable to model systems with dynamic solid objects. Having the feature to add movable walls would allow for simulation of interactions between solid physical objects and gasses. The project will begin with simple moving walls that will influence particle dynamics, but the particles will not influence the walls dynamics. Afterwards, implementing wall momentum dependent on particle collisions will be developed. There are some questions on to compensate for limited particle velocities(-1,0,+1) in the LG2d simulation. When a wall and a particle collide and they have opposite velocities, how does the particle reflect and keep momentum conserved? The particle currently can't have a velocity magnitude greater than 1. Perhaps the LGd2 simulation would require the feature to allow for higher particle velocities.
\section{Research Group}
There are several ways of dividing the labor for this project. Part of the group could work on the particle collision independent moving walls and the other group could work on collision dependent moving walls. This would divide the time line below into 2 time lines. One issue is that developing the dependent walls may be more difficult until the independent walls are completed. 
\section{Milestones}
This project will depend on the a previously developed integer lattice gas simulation: "LG2d.c". This simulation already has features similar to the ones proposed in this project such as static walls that reflect particles. Code will be added to the LG2d.c simulation to observe how particles behave when the wall is dynamic instead of static walls. 
\begin{itemize}
  \item Simulate with stationary wall and particles initially moving uniformly perpendicular to the wall. Observe how the wall influences the particles' average velocity.  
  \item Develope code to simulate a moving wall(independent of particle collisions). Simulate with initailly stationary particles. Observe how the wall influences particles average velocity. Compare results to item 1 on this list. They should have a similar result because both particles and wall are both moving with the same velocity relative to eachother.
  \item Develope code to simulate a stationary tube. Place two dynamic walls on each end of the tube with particles inside this chamber. Move walls inward. Verify particles do not leak out. Observe how the particle density changes as the chamber decreases in size. Observe the forces on the walls as the chamber size changes.
  \item Develope code to make wall momentum to be dependent on particle collisions. Simulate a hollow, hoizontal, stationary tube with one horizontally movable, vertical wall. place gasses in each side of the vertical wall with different densities. Observe the movement of the movable wall. The wall should stop moving when both particle desities equalize.\newline
\vspace{5mm}\newline
Additional long term milestones could include the addition of rotating walls, hinges to connect walls, flexible walls. These features may be unrealistic with the avaliable time, but coule prove to be useful in creating more complex simulations.
\end{itemize}
\section{Timeline}
\begin{itemize}
  \item Tuesday, April 17th: Have milestone 1 and 2 completed
  \item Thursday, April 19th: Have milestone 3 started (even if leaking particles)
  \item Tuesday, April 24th: Have milestone 3 completed and 4 started. 
  \item Thursday, April 26th: Have 4 completed.
  \item Tuesday, May 1st: Have 5 started/partially functional.
  \item Thursday, May 3rd: Have 5 completed.
  \item Tuesday, May 8th: Have results and project deliverables compiled and ready to turn in.
\end{itemize}
\end{document}



  
